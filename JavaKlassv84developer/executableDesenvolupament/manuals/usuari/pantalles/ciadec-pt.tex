
En esta segunda parte del proyecto de tesis, se hace una breve
introducci\'on al sistema CIADEC~\footnote{de las primeras letras
del nombre Caracterizaci\'on e Interpretaci\'on Autom\'atica de
Descripciones Conceptuales.}, se describe su estructura y sus
funcionalidades.

\section{\bf Introducci\'on}

El sistema CIADEC implementa la metodolog\'{\i}a que se defini\'o
en~\cite{vaz:gib:02} cuyo t\'{\i}tulo es ``Automatic Generation of
Fuzzy Rules in ill-Structured Domains"~\cite{vaz:gib:01}, la que
permite caracterizar las diferentes clases a partir de una
clasificaci\'on previamente establecida, en dominios poco
estructurados y obtener autom\'aticamente interpretaciones
conceptuales de \'estas, con respecto a atributos cuantitativos.

\section{\bf Dise\~{n}o modular del sistema CIADEC}

El sistema CIADEC, surge de la necesidad de automatizar la
caracterizaci\'on e interpretaci\'on de clases en dominios poco
estructurados previamente particionados combinando conceptos,
t\'ecnicas de inteligencia artificial y estad\'{\i}stica. Mediante
la automatizaci\'on se persigue reducir el tiempo necesario para
llevar a cabo esta tarea, agilizando tanto las actividades
asociadas al an\'alisis de datos como a la obtenci\'on de
informaci\'on relevante que posteriormente sea \'util en la
gesti\'on y/o toma de decisiones en esos dominios.

\subsection{\bf Arquitectura del sistema CIADEC}

La entrada del sistema es la matriz de datos $X$ y la partici\'on
de referencia $P$, teniendo como salidas, seg\'un la opci\'on del
usuario:

\begin{itemize}
\item La asignaci\'on de clases a un conjunto de objetos nuevos.
\item La calidad de asignaci\'on del sistema de reglas.
\item La representaci\'on gr\'afica de las funciones de pertenencia
por atributo. Dichos gr\'aficos son generados en c\'odigo \LaTeX\
y se pueden exportar a cualquier documento o bien ser visualizados
en pantalla conectando con el visualizador de \LaTeX.  Este
tratamiento se adec\'ua a la filosof\'{\i}a de otras herramientas
que se comparten en el mismo equipo de trabajo y que, en un
futuro, se han de integrar en una plataforma com\'un.
\end{itemize}

A nivel conceptual el sistema CIADEC esta formado por los
siguientes cinco m\'odulos:

\begin{itemize}
    \item M\'odulo I. Generador de Intervalos de Longitud Variable (GILOVA)
    \item M\'odulo II. Generador de Tablas de Distribuciones (Funciones de
            Pertenencia) Condicionadas a  Intervalos (GETADI)
    \item M\'odulo III. Generador de Sistemas de Reglas (GESIRE)
    \item M\'odulo IV. Generador de Gr\'aficos de funciones de pertenencia
     de $X_{k}|C$ (GEGRALA)
    \item M\'odulo V. Validaci\'on (VALIDA)
\end{itemize}

\begin{figure}
\begin{center}
\scalebox{0.8}{\includegraphics{bien}}
%%\mbox{}
%%\vspace{-4 cm}
\caption{\label{fig:uno}}{Diagrama Conceptual del Proceso CIADEC}
\end{center}
\end{figure}

\subsection{\bf Estructuras de datos}
En este apartado se explica la representaci\'on de datos y la
estructura de ficheros que el sistema CIADEC necesita para que
funcione.

\subsubsection{Representaci\'on de datos}

Los individuos que forman el conjunto $T_{0}$ est\'an descritos
por una serie de atributos o caracter\'{\i}sticas y pueden ser de
dos tipos:
\begin{itemize}
    \item Atributos cualitativos o categ\'oricos: Corresponden a un tipo de
    caracter\'{\i}stica de los individuos que se expresan mediante
    adjetivos. Estos atributos cualitativos se dividen en
    ordinales~\footnote{Son aquellas para las que existe una relaci\'on de
    orden} y nominales~\footnote{Son aquellas que no presentan una
    ordenaci\'on de sus valores}.
    \item Atributos cuantitativos: Son caracter\'{\i}sticas
    medibles y se expresan en forma num\'erica.
\end{itemize}
Si se dispone de $n$ individuos y de $k$ atributos que los
describen, los valores de todas estas variables para el conjunto
de individuos se representan mediante una matriz rectangular $X$
de dimensiones ($n$, $k$). Las filas de la matriz contendr\'an la
informaci\'on de los individuos, mientras que las columnas hacen
referencia a los atributos. Si los individuos son caracterizados
simult\'aneamente con atributos cuantitativos y cualitativos, la
matriz de datos se considera heterog\'enea.

A las observaciones no presentes en la matriz de datos $X$ se les
denomina valores faltantes. En caso de valores faltantes les
asignamos un $``*"$ con valor NaN (Not a Number) para que sea
tratable desde el punto de vista del algoritmo.

La Figura~\ref{fig:uno} muestra la arquitectura modular del
sistema CIADEC.

\subsection{\bf Estructuras de ficheros}
En este apartado se la estructura de ficheros que el sistema
CIADEC necesita para que funcione.

Las estructuras de los ficheros que describen el flujo de datos en
el sistema CIADEC son de dos tipos de entrada y salida y
extensiones: dat, par, iks, dci, tex, srg, srr, tcp, vsg, vsm y
coi.

\begin{itemize}
    \item $<nombre\_fichero.dat>$\ \ Contiene la matriz de
    datos $X$ por renglones. Para cada individuo u objeto $i$ hay una
    lista con las coordenadas que le definen en cada atributo y su
    formato es el est\'andar de este tipo de fichero: Los elementos
    de una l\'{\i}nea estar\'an separados por al menos un espacio. En la
    primera l\'{\i}nea van los nombres de los atributos, en caso
    de que no est\'en los nombres se asignar\'an a los atributos
    los nombres por defecto NONAMEk, donde k es el n\'umero del
    atributo en consideraci\'on. El Cuadro~\ref{uno} muestra el
    formato de un fichero con extensi\'on dat.
\begin{table}[h]
\begin{center}
    \begin{tabular}{lrrrrl}
   $x_{1}$& $x_{2}$ & \ldots & \ldots &$x_{n}$ & xx
     \\
     $v_{11}$&$v_{12}$ & \ldots & \ldots & $v_{1n}$ & $id_{1}$  \\
      $v_{21}$&$v_{22}$ & \ldots & \ldots & $v_{2n}$ & $id_{2}$ \\
    \ldots  & \ldots  & \ldots  & \ldots  & \ldots & \ldots \\
    \ldots  & \ldots & \ldots & \ldots & \ldots & \ldots  \\
     $v_{m1}$&$v_{m2}$ & \ldots & \ldots & $v_{mn}$ & $id_{m}$
    \end{tabular}
\caption{Estructura de un fichero extensi\'on dat}
   \label{uno}
\end{center}
\end{table}
%\vspace{-1cm}
\item $<nombre\_fichero.par>$\ \ Cuando la partici\'on de referencia no est\'a
inclu\'{\i}da  en la matriz de datos $X$, este fichero contiene
informaci\'on referida a dicha partici\'on, su estructura es una
columna que conserva el orden de asignaci\'on de la clase de
referencia con respecto al orden de los individuos en el conjunto
de datos de la matriz $X$. El Cuadro~\ref{luna} muestra el formato
de un fichero con extensi\'on par.

\begin{table}[h]
\begin{center}
\begin{tabular}{cc}
 nom\_Obj &  Clase \\
 nom\_1 &  $id_{1}$  \\
 nom\_2 &    $id_{2}$  \\
 \ldots & \ldots \\
 \ldots & \ldots \\
  \ldots & \ldots \\
nom\_n &     $id_{n}$
\end{tabular}
\caption{Estructura de un fichero extensi\'on par}
   \label{luna}
\end{center}
\end{table}

\item $<nombre\_fichero.iks>$\ \ Un fichero con extensi\'on iks
contiene informaci\'on sobre el sistema de intervalos de longitud
variable correspondiente a el atributo $X_{k}$, su estructura
consiste de un rengl\'on donde se encuentran los $2\xi$ valores
l\'{\i}mites del sistema de intervalos separados por al menos un
espacio. El Cuadro~\ref{unico} muestra el formato de un fichero
con extensi\'on iks.

\begin{table}[h]
\begin{center}
\begin{tabular}{ccccccc}
$Z_{1}$ & $Z_{2}$ & $Z_{3}$ & \ldots & \ldots & $Z_{2\xi-1}$ &
$Z_{2\xi}$
\end{tabular}
\caption{Estructura de un fichero extensi\'on iks}
   \label{unico}
\end{center}
\end{table}

\item $<nombre\_fichero.dci>$\ \ Un fichero con extensi\'on dci
      contiene informaci\'on sobre la tabla de distribuciones condicionadas
      a intervalos para un cierto atributo $X_{k}$ cuya estructura es la
      siguiente:  en la primera l\'{\i}nea van los l\'{\i}mites de los
      intervalos y en el resto de las casillas los valores de la funci\'on de
      pertenencia $p_{sc}$ por clase $C$, todos sus elementos est\'an separados
      por al menos un espacio. El Cuadro~\ref{dos} muestra el formato de
      un fichero con extensi\'on dci.

\begin{table}[h]
\begin{center}
    \begin{tabular}{ccccccc}
%%$Z_{1}$ & $Z_{2}$ & $Z_{3}$ & \ldots & \ldots & $Z_{2\xi-1}$ &
%%$Z_{2\xi}$ \\
$p_{11}$ &$p_{21}$ &$p_{31}$ &\ldots &\ldots
&$p_{(2\xi-1)1}$ & \\
$p_{12}$ &$p_{22}$ &$p_{32}$ &\ldots &\ldots
&$p_{(2\xi-1)2}$ &  \\
\ldots & \ldots & \ldots & \ldots & \ldots & \ldots &  \\
$p_{1\xi}$ &$p_{2\xi}$ &$p_{3\xi}$ &\ldots &\ldots
&$p_{(2\xi-1)\xi}$ &
    \end{tabular}
\caption{Estructura de un fichero extensi\'on dci}
   \label{dos}
\end{center}
\end{table}

%\vspace{-1cm}
\item $<nombre\_fichero.tex>$\ \ Fichero que contiene la
      estructura de las instrucciones en c\'odigo \LaTeX\ de
      los gr\'aficos de las funciones de pertenencia condicionadas
      a intervalos por atributo $X_{k}$ y por clase $C$ . El
      Cuadro~\ref{tres} muestra el formato de un fichero con
      extensi\'on tex.

\begin{table}[h]
\begin{center}
    \begin{tabular}{l}
 \% Contenido de la \emph{figura grande} para el atributo $X_{k}$\\
       $\backslash begin\{figure\}$ \\
       \{$\backslash setlength\
       \{\backslash unitlength \{1pt\}$\} \\
       $\backslash begin\{picture\} (540, 700) (50, -175)$ \\
        \% Para cada clase $Ci$ el origen de cada grafo se coloca en la posici\'on \\
        \% (0, $140 \cdot c$), donde $c$ var\'{\i}a de 0 a 3 en cada hoja
        tama\~no\\
         a4, utilizando la instrucci\'on: \\
       $\backslash put(0, 140 \cdot c) \{G_{C}\}$ \\
        donde $G_{C}$ es el c\'odigo \LaTeX\ del grafo para la clase $C$, con
        $1 \leq $C$ \leq \xi$. \\
       $\backslash end\{picture\}$ \\
       $\backslash end\{figure\}$
       \end{tabular}
\caption{Estructura de un fichero extensi\'on tex}\label{tres}
\end{center}
\end{table}

\item $<nombre\_fichero.srg>$\ \ Fichero que contiene la
estructura del sistema de reglas globales $\mathcal{\Re}(X_{k},
\mathcal{P})$ para el atributo seleccionado $X_{k}$, el contenido
de este fichero es recomendable sea tipo latex. Es un fichero en
forma de columna de $(2\xi-1)(\xi)$ elementos, el sub-\'{\i}ndice
de la regla nos marca la posici\'on en que se dispara la regla. El
Cuadro~\ref{mar} muestra el formato de un fichero con extensi\'on
srg.

\begin{table}[h]
\begin{center}
    \begin{tabular}{c}
$ r_{11}:\ x_{ik}\in
I_{1}^{k}\stackrel{p_{sc}}{\longrightarrow}C1$ \\
$ r_{12}:\ x_{ik}\in
I_{1}^{k}\stackrel{p_{sc}}{\longrightarrow}C1$ \\
\ldots \\
\ldots \\
$ r_{2\xi-1,\xi}:\ x_{ik}\in
I_{2\xi-1}^{k}\stackrel{p_{sc}}{\longrightarrow}C\xi$
    \end{tabular}
\caption{Estructura de un fichero extensi\'on srg}
   \label{mar}
\end{center}
\end{table}

\item $<nombre\_fichero.srr>$\ \ Fichero que contiene la
estructura de un sistema reducido de reglas
$\mathcal{\Re}^{*}(X_{k}, \mathcal{P})$ para el atributo
seleccionada $X_{k}$ cuando se ha optado por escoger alg\'un
criterio de agregaci\'on. Como una primera aproximaci\'on en este
trabajo hemos elegido el criterio de m\'axima probabilidad,
as\'{\i} que tenemos una regla por cada intervalo $I_{s}^{k}$ del
sistema $\mathcal{I}^{k}$. El Cuadro~\ref{sol} muestra el formato
de un fichero con extensi\'on srr.

\begin{table}[h]
\begin{center}
    \begin{tabular}{c}
$ r_{1}:\ x_{ik}\in
I_{1}^{k}\stackrel{p_{max}}{\longrightarrow}C$ \\
%
$ r_{2}:\ x_{ik}\in
I_{2}^{k}\stackrel{p_{max}}{\longrightarrow}C$ \\
%
\ldots  \\
\ldots  \\
%
$ r_{2\xi-1}:\ x_{ik}\in
I_{2\xi-1}^{k}\stackrel{p_{max}}{\longrightarrow}C$
    \end{tabular}
\caption{Estructura de un fichero extensi\'on srr}
   \label{sol}
\end{center}
\end{table}

\item $<nombre\_fichero.tcp>$\ \ Fichero que contiene
informaci\'on sobre la comparaci\'on entre la clasificaci\'on de
referencia ($C$) y la obtenida por el sistema de reglas
($\hat{C}$) en cada atributo, donde $c_{ij}$ es el n\'umero de
coincidencias entre ambas clasificaciones para el atributo
$X_{k}$. El Cuadro~\ref{venus} muestra el formato de un fichero
con extensi\'on tcp.

\begin{table}[h]
\begin{center}
    \begin{tabular}{lrrrl}
         & $C_{1}$ &  \ldots & \ldots &$C_{\xi}$
     \\
     $\hat{C1}$ & $c_{11}$ & \ldots & \ldots & $c_{1\xi}$  \\
      $\hat{C2}$&$c_{21}$ & \ldots & \ldots & $c_{2\xi}$  \\
    \ldots  & \ldots  & \ldots  & \ldots  & \ldots  \\
    \ldots  & \ldots & \ldots & \ldots & \ldots  \\
     $\hat{C}\xi$&$c_{\xi 1}$ & \ldots & \ldots & $c_{\xi \xi}$
    \end{tabular}
\caption{Estructura de un fichero extensi\'on tcp}
   \label{venus}
\end{center}
\end{table}

\item $<nombre\_fichero.vsg>$\ \ Fichero que contiene la
informaci\'on de las probabilidades y consecuentes de las reglas
que se dispar\'an para los individuos del conjunto de prueba
$P_{0}$. El Cuadro~\ref{vale} muestra el formato de este tipo de
fichero.

\begin{table}[h]
\begin{center}
    \begin{tabular}{lrrrrl}
    No.  & $C$ & $P$ & \ldots & $C$ & $P$ \\
     1 & $C_{11}$ & $p_{11}$ & \ldots & $C_{1\xi}$ & $p_{1\xi}$
     \\
     2 & $C_{21}$ & $p_{21}$ & \ldots & $C_{2\xi}$ & $p_{2\xi}$  \\
    \ldots  & \ldots  & \ldots  & \ldots  & \ldots & \ldots \\
    \ldots  & \ldots & \ldots & \ldots & \ldots & \ldots \\
     n &$C_{n 1}$ & $p_{n1}$ & \ldots & $c_{n \xi}$ & $p_{n\xi}$
    \end{tabular}
\caption{Estructura de un fichero extensi\'on vsg}
   \label{vale}
\end{center}
\end{table}
\item $<nombre\_fichero.vsm>$\ \ Fichero que contiene la
informaci\'on de las probabilidades m\'aximas y consecuentes de
las reglas que se dispar\'an para los individuos del conjunto de
prueba $P_{0}$. El Cuadro~\ref{cero} muestra el formato de este
tipo de fichero.
\begin{table}[h]
\begin{center}
    \begin{tabular}{lcr}
    No.  & $C$ & $P_{max}$  \\
     1 & $C_{1}$ & $p_{1max}$
     \\
     2 & $C_{2}$ & $p_{2max}$  \\
    \ldots  & \ldots  & \ldots   \\
    \ldots  & \ldots & \ldots  \\
     n &$C_{n}$ & $p_{nmax}$
    \end{tabular}
\caption{Estructura de un fichero extensi\'on vsm}
   \label{cero}
\end{center}
\end{table}
\item $<nombre\_fichero.coi>$\ \ Fichero que contiene la
informaci\'on sobre la coincidencias entre las clases de
predicci\'on ($\hat{C}$) y la referencia ($C$) para cada uno de
los individuos del conjunto de prueba. El Cuadro~\ref{pila}
muestra la estructura de este tipo de fichero con extenci\'on coi.
\begin{table}[h]
\begin{center}
    \begin{tabular}{lcr}
    No.  & $\hat{C}$ & $C$  \\
     1 & $\hat{C}_{1}$ & $C_{1}$
     \\
     2 & $\hat{C}_{2}$ & $C_{2}$  \\
    \ldots  & \ldots  & \ldots   \\
    \ldots  & \ldots & \ldots  \\
     n &$\hat{C}_{n}$ & $C_{n}$ \\
     N\'umero &  de &coincidencias:
    \end{tabular}
\caption{Estructura de un fichero extensi\'on coi}
   \label{pila}
\end{center}
\end{table}
\end{itemize}

\subsection{\bf Descripci\'on de los m\'odulos del sistema CIADEC}

La descripci\'on de los m\'odulos que forman la aquitectura de
CIADEC se enfoca sobre el objetivo principal, la entrada, salida y
expectativas de funcionamiento de cada uno de ellos.

\subsubsection{M\'odulo I: Generaci\'on de Intervalos de Longitud
Variable (GILOVA)} Dada el atributo de estudio $X_{k}$ el m\'odulo
I tiene como principal objetivo el de generar un sistema de
intervalos de longitud variable calculando los valores de
$\mathcal{I}^{k}$. La llamada a la funci\'on principal del
m\'odulo es:
\begin{center}
    TablaInterv($X$, $X_{k}$, $P$, int nclases)
\end{center}
\paragraph{Entrada:} Los par\'ametros de entrada en este m\'odulo son:
el atributo a seleccionar $X_{k}$, la matriz de datos $X$ y en su
caso la partici\'on $P$ del conjunto de datos y el n\'umero de
clases nclases de la partici\'on.
\paragraph{Salida:}
Para cada atributo seleccionado $X_{k}$, como salida un vector que
representa el sistema de intervalos de longitud variable y cuya
estructura es la de un fichero con extensi\'on iks, digamos el
nombre del atributo en estudio y representado por $<X_{k}.iks>$.

\paragraph{Descripci\'on:}
En este m\'odulo se realizan las siguientes funciones al hacer la
llamada a la funci\'on principal y son: Construir un vector con el
m\'{\i}nimo y m\'aximo de $X_{k}$ en cada clase, ordenar los
valores de ese vector de menor a mayor y generar el fichero .iks.

\subsubsection{M\'odulo II: Generaci\'on de la Tabla de Distribuciones
Condicionadas a Intervalos (GETADI)}

El m\'odulo II GETADI tiene como principal objetivo la
generaci\'on de la tabla de distribuciones condicionadas a
intervalos (o funciones de pertenencia). La llamada a la funci\'on
principal del m\'odulo es:
\begin{center}
    TablaFrec($X$, $X_{k}$, $P$, short nClases, short nInterv)
\end{center}
\paragraph{Entrada:} Este m\'odulo tiene como entrada el conjunto de datos $X$,
el atributo seleccionado $X_{k}$, la partici\'on $P$, el n\'umero
de clases nClases y el n\'umero de intervalos nInterv.

\paragraph{Salida:} La salida en este m\'odulo es una tabla de
distribuciones condicionada a intervalos de la forma
$\mathcal{P}|I^{k}$ representada por medio de un fichero el nombre
del atributo y con extensi\'on dci ($<X_{k}.dci>$).

\paragraph{Descripci\'on:} En esta parte se lee nuevamente la columna
$X_{k}$ de la matriz de datos $X$, ubicando cada valor $x_{ik}$ en
el intervalo $\mathcal{I}_{s}^{k}$ y clase $C$ correspondiente,
contando el n\'umero de elementos en cada una de las casillas de
la tabla $\mathcal{P}|I^{k}$ desde $k=1 : 2\xi-1$ y posteriormente
sumando columnas obtenemos los $n_{I_{s}^{k}}$ para dividir cada
casilla por su $n_{I_{s}^{k}}$ y determinar las probabilidades
$p_{sc}$ correspondientes.

\subsubsection{M\'odulo III:  Generaci\'on de Sistemas de Reglas (GESIRE)}

El m\'odulo III GESIRE genera primero, un sistema de reglas
difusas de inducci\'on basado en la matriz de distribuciones
condicionada a intervalos por atributo seleccionado $X_{k}$,
$\mathcal{\Re}(X_{k}, \mathcal{P})$ en un fichero de tipo .srg
para el sistema de reglas completo y una vez que hemos elegido un
cierto criterio de agregaci\'on para reducir la ambiguedad
inherente al sistema de reglas obtenemos un sistema de reglas
reducido $\mathcal{\Re}^{*}(X_{k},\mathcal{P})$ que se guarda en
un fichero tipo .srr. La llamada a la funci\'on principal del
m\'odulo es:
\begin{center}
    GeneradorReglas($<X_{k}.dci>$, $X_{k}$, short nClases, short nInterv)
\end{center}
\paragraph{Entrada:} La entrada es un fichero con extensi\'on
.dci que representan la tabla $\mathcal{P}|I^{k}$ de
distribuciones condicionada a intervalos del atributo $X_{k}$ en
estudio y sus dimensiones nClases y nInterv.
\paragraph{Salida:} La salida es a dos niveles dependiendo del inter\'es
del usuario: si se desea el conjunto global de reglas es un
fichero con extensi\'on src, por otro lado si se desea elegir
alg\'un criterio de agregaci\'on la salida ser\'a un fichero con
extensi\'on srr.

\paragraph{Descripci\'on:} En este m\'odulo se construye el
sistema global de reglas $\mathcal{\Re}(X_{k}, \mathcal{P})$ por
atributo seleccionado $X_{k}$ a partir de la tabla  de
distribuciones condicionada a intervalos y considerando un
criterio de agregaci\'on (pj., el de probabilidad m\'axima)
obtenemos un sistema reducido de reglas $\mathcal{\Re}^{*}(X_{k},
\mathcal{P})$.

\subsubsection{M\'odulo IV:  Generaci\'on de Gr\'aficos en
\LaTeX\ (GEGRALA)}

El m\'odulo IV GEGRALA tiene como objetivo generar gr\'aficos para
las funciones de pertenencia $f(X_{k}|C)$. La llamada a la
funci\'on principal del m\'odulo es:
\begin{center}
    GeneradorGrafico($<X_{k}.dci>, X_{k}.tex, X_{k}$)
\end{center}
\paragraph{Entrada:} El fichero de entrada para la generaci\'on
de gr\'aficos \LaTeX\ es la tabla $\mathcal{P}|I^{k}$
representadas por un fichero con extensi\'on dci ($<X_{k}.dci>$).

\paragraph{Salida:} La salida es de dos tipos:
    \begin{itemize}
    \item Como fichero con extensi\'on .tex, con el nombre del atributo de estudio
($<nombre\_Atributo.tex>$).
    \item Y visualizaci\'on en pantalla del gr�fico.
    \end{itemize}
La estructura de los ficheros tex es la que sigue un formato de
 fichero en \LaTeX\ (ver Cuadro~\ref{tres}) que dibuje el gr�fico.

\paragraph{Descripci\'on:} En este m\'odulo la generaci\'on
de gr\'aficos de las tablas $\mathcal{P}|I^{k}$ de distribuciones
se hace a partir de la tabla de distribuciones generada en el
m\'odulo II y considerando las operaciones de transfomaci\'on en
\ref{gengra}.

\subsubsection{M\'odulo V:  Validaci\'on del Sistema de Reglas (VALIDA)}

Este m\'odulo representa la etapa final del proceso y es donde se
realiza la validaci\'on del sistema de reglas que hemos obtenido
cuando existe un conjunto de validaci\'on $P_{0}$, primero para
cada una de los atributos $X_{k}$ y luego una vez hecha la
asignaci\'on de la clase correspondiente $C$ a cada uno de los
individuos $i$ del conjunto de prueba $P_{0}$, compar�ndola con la
clase de referencia asociada a cada uno de \'estos obtener la
calidad de asignaci\'on del sistema. La llamada a la funci\'on
principal del m\'odulo es:

\begin{center}
    ValidaReglas(FileDat $P_{0}.dat$, short nInterv)
\end{center}

\paragraph{Entrada:} Recibe como entrada el conjunto de
prueba $P_{0}$ donde cada individuo tiene asignada la clase ($C$)
que le corresponde de acuerdo a la clasificaci\'on de referencia y
por otro lado, la tabla $\mathcal{I}^{k}|P$ ($<X_{k}.dci>$). Con
esta tabla y el valor $x_{ik}$ se puede calcular la  clase de
predicci\'on ($\hat{C}$) de cada individuo y aplicando el criterio
de probabilidad m\'axima obtener el sistema reducido de reglas.

\paragraph{Salida:} La salida es un fichero con extensi\'on tcp
que determina el n\'umero de coincidencias entre las clases de
referencia ($C$) y de predicci\'on ($\hat{C}$) para los elementos
de $P_{0}$.

\paragraph{Descripci\'on:} En este m\'odulo se hace una
comparaci\'on cruzada entre la clase asignada (la de referencia) y
la de predicci\'on (la asignada debido al sistema de reglas
reducido) para determinar el grado de confiabilidad de nuestro
sistema de reglas. Se calcula en \% de error.

\subsection{\bf Sobre la Generaci\'on de Gr\'aficos en \LaTeX }
\label{gengra}

En esta parte se propone una forma de representar gr\'aficamente
este sistema de reglas que permite obtener conocimiento \'util y
comprensible para la interpretaci\'on conceptual de las clases
identificadas.

A nivel dise�o este m\'odulo se hizo en forma imbricada a dos
niveles para la reutilizaci\'on del c\'odigo \LaTeX\ de estos
gr\'aficos en documentos posteriores.
\input{disegrafo.tex}
Utilizando la misma forma imbricada de figuras, el otro tipo de
gr\'afico en este m\'odulo, el de gr\'aficos que representan la
combinaci\'on de atributos se hizo de la siguiente forma:
Considerando como entrada el conjunto de prueba $P_{0}$, se lee
para el primer individuo $i=1$ el valor de la primer atributo
$X_{1}$, se localiza el intervalo que le corresponde en la matriz
de distribuciones condicionadas a intervalos de ese atributo y se
toma ese rengl\'on con sus clases y probabilidades
correspondientes y se construye el primer gr�fico ; luego se lee
el segundo atributo $X_{2}$ se localiza el intervalo a que
corresponde en la matriz de distribuciones de ese atributo y se
construye este gr�fico, que representar� todas las clases
asociadas con sus correspondientes probabilidades y as�
sucesivamente hasta el atributo $X_{17}$, de todo esto obtenemos
17 gr�ficos para el primer individuo. En seguida se considera el
segundo individuo $i=2$ y se repite el proceso anterior y as�
sucesivamente hasta agotar todos los individuos del conjunto de
prueba (30 elementos), de tal forma que obtengamos 17 gr�ficos que
representan las clases y sus correspondientes probabilidades de
los 17 atributos seleccionadas por individuos en el conjunto de
prueba $P_{0}$.

A nivel algoritmo se definen dos constructores: Class
GeneradorGrafico y Class GeneradorLatex.

\section{\bf Implementaci\'on del sistema CIADEC}

CIADEC es un Sistema orientado a la caracterizaci\'on e
interpretaci\'on auto\-m\'atica de clases en dominios poco
estructurados implementado en el lenguaje de programaci\'on JAVA 2
SDK, versi\'on $1.3.1\_01$ en la plataforma Windows 95/ 98/ 2000/
NT4.0. El sistema deber\'a operar en un entorno de PC\'\ s y se ha
desarrollado a partir de la metodolog\'{\i}a formal descrita en el
ya citado reporte~\cite{vaz:gib:01}.

\end{document}
